\tikzfeynmanset{
    every vertex={dot}, % dot
}
\begin{tikzpicture} [baseline={(current bounding box.center)}]
    \begin{feynman}
        \vertex (a);
        \vertex [above left=of a] (b) {\Pquark};
        \vertex [below left=of a] (d) {\APquark};
        \vertex [dot, right=of a] (f);
        \vertex [particle, above right=of f] (c);
        \vertex [below right=of f] (e) {\APbottom};
        \vertex [above right=of c] (g) {\Pbottom};
        \vertex [below right=of c] (h);
        \vertex [above right=of h] (i) {\Ppositron};
        \vertex [below right=of h] (j) {\Pelectronneutrino};
        \diagram* {
            (b)[dot] -- [fermion] (a) -- [fermion] (d),
            (a) -- [boson, edge label'=\PWplus] (f),
            (e) -- [fermion] (f) -- [fermion, edge label=\Ptop] (c),
            (c) -- [fermion] (g),
            (c) -- [boson, edge label=\PWplus] (h),
            (h) -- [anti fermion] (i),
            (h) -- [fermion] (j),
        };
    \end{feynman}
\end{tikzpicture}