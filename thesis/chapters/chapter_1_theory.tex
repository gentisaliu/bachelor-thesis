This chapter discusses the relevant theory for the thesis. Section \ref{sec:theory_sm} provides an overview of the experimental aspects of the standard model (SM) of particle physics, the context of this work. Section \ref{sec:theory_top} focuses on properties of the top quark, its production processes and decay modes, which will be relevant for the development and evaluation of the classifier in chapters \ref{ch:classifier} and \ref{ch:classifier}.
\section{The Standard Model of Particle Physics}
\label{sec:theory_sm}
The SM is a quantum field theory that describes the fundamental forces in the universe and classifies all known elementary particles. It includes quantum mechanics, special relativity and chromodynamics. Underlying interactions of matter are described by the $\textrm{U}(1)_\textrm{Y}\otimes \textrm{SU}(2)_\textrm{L}\otimes \textrm{SU}(3)_\textrm{C}$ symmetry group. It is the result of many theoretical and experimental studies, which were conducted throughout the latter half of the 20th century and is in excellent accord with almost all current data \cite{RevModPhys.71.S96}.\\ \\
Particles in the SM are divided in fermions and bosons based on their spin quantum number.\\ \\
\textbf{Fermions} are half-integer spin particles that follow Fermi-Dirac statistics and obey the Pauli exclusion principle. They are classified into six quarks (up, down, charm, strange, top, bottom) and six leptons (electron, electron neutrino, muon, muon neutrino, tau, tau neutrino). Pairs from each classification form a generation. A summary is given in table \ref{tab:ch_1_sm_fermions}.
\begin{table}[h]
\caption[Fermions of the SM]{Fermions of the standard model. Source: \cite{povh}}
\label{tab:ch_1_sm_fermions}
\begin{tabular}{ccccccccc}
\toprule
\multirow{2}{*}{Fermions} & \multicolumn{3}{c}{Generation} & {Electric} & \multirow{2}{*}{Color} & \multicolumn{2}{c}{$3^{\mathrm{rd}}$ component of isospin $I_3$} & \multirow{2}{*}{Spin}\\
& 1 & 2 & 3 & {charge} & & left-handed & right-handed & \\
\midrule
\multirow{2}{*}{Leptons} & $\upnu_\textrm{e}$ & $\upnu_\upmu$ & $\upnu_\uptau$ & {$0$} & \multirow{2}{*}{---} & \multirow{2}{*}{$\nicefrac{1}{2}$} & --- & \multirow{2}{*}{$\nicefrac{1}{2}$}\\
& e & $\upmu$ & $\uptau$ & {$-1$} & & & {$0$} &\\
\midrule
\multirow{2}{*}{Quarks} & u & c & t & $+\nicefrac{2}{3}$ & \multirow{2}{*}{r, b, g} & \multirow{2}{*}{$\nicefrac{1}{2}$} & {$0$} & \multirow{2}{*}{$\nicefrac{1}{2}$}\\
& d & s & b & $-\nicefrac{1}{3}$ & & & {$0$} & \\
\bottomrule
\end{tabular}
\end{table}
Quarks carry color charge, hence they interact via the strong interaction. Color confinement leads to quarks being strongly bound to one another, forming color-neutral composite particles (hadrons) that contain either a quark and an antiquark (mesons) or three quarks (baryons, such as protons and neutrons). Quarks also carry electric charge and weak isospin, thus they interact with other fermions both electromagnetically and via the weak interaction. Consequently, quarks of the same generation form a weak isospin doublet, where they behave similarly towards the weak interaction.

The remaining six fermions do not carry color charge and are called leptons. The three neutrinos do not carry electric charge either, their motion is directly influenced only by the weak nuclear force, which makes them difficult to detect. Since they carry an electric charge, the electron, muon, and tau all interact electromagnetically.

Particles of a generation have greater mass than those of lower generations. The first-generation charged particles do not decay, all ordinary matter is made of such particles. Second- and third-generation charged particles decay with very short half-lives and are observed only in high-energy environments. Neutrinos of all generations also do not decay and pervade the universe, but rarely interact with baryonic matter.\\ \\
\textbf{Bosons} are defined as force carriers that mediate the strong, weak, and electromagnetic fundamental interactions.

Interactions in physics are the ways that particles influence other particles. At a macroscopic level, electromagnetism allows particles to interact with one another via electric and magnetic fields, and gravitation allows particles with mass to attract one another in accordance with Einstein's theory of general relativity. The Standard Model explains such forces as resulting from matter particles exchanging other particles, generally referred to as force mediating particles. When a force-mediating particle is exchanged, at a macroscopic level the effect is equivalent to a force influencing both of them, and the particle is therefore said to have mediated (i.e., been the agent of) that force. The Feynman diagram calculations, which are a graphical representation of the perturbation theory approximation, invoke "force mediating particles", and when applied to analyze high-energy scattering experiments are in reasonable agreement with the data. However, perturbation theory (and with it the concept of a "force-mediating particle") fails in other situations. These include low-energy quantum chromodynamics, bound states, and solitons.
\begin{table}[h]
	\caption[Bosons described by the SM]{The bosons of the standard model. Source: \cite{povh}, \cite{faltermann}}
	\label{tab:ch_1_sm_bosons}
	\begin{tabular}{cccccc}
		\toprule
		Interaction & Acts on & Carrier of the force & Mass (GeV) & {$J^P$} & Range (m)\\
		\midrule
		strong & color charge & 8 gluons (g) & {$0$} & {$1^-$} & $\approx 10^{-15}$\\
		electromagnetic & electric charge & Photon ($\upgamma$) & {$0$} & {$1^-$} & $10^{-15}$\\
		weak & weak charge & W$^{\pm}$, Z$^0$ & {$80.385$, $91.188$} & {$1$} & $\infty$\\
		\bottomrule
	\end{tabular}
\end{table}
The gauge bosons of the Standard Model all have spin (as do matter particles). The value of the spin is 1, making them bosons. As a result, they do not follow the Pauli exclusion principle that constrains fermions: thus bosons (e.g. photons) do not have a theoretical limit on their spatial density (number per volume). The different types of gauge bosons are described below.

Photons mediate the electromagnetic force between electrically charged particles. The photon is massless and is well-described by the theory of quantum electrodynamics.
The W+, W-, and Z gauge bosons mediate the weak interactions between particles of different flavors (all quarks and leptons). They are massive, with the Z being more massive than the $\textrm{W}^{\pm}$. The weak interactions involving the W± exclusively act on left-handed particles and right-handed antiparticles. Furthermore, the $\textrm{W}^{\pm}$ carries an electric charge of +1 and -1 and couples to the electromagnetic interaction. The electrically neutral Z boson interacts with both left-handed particles and antiparticles. These three gauge bosons along with the photons are grouped together, as collectively mediating the electroweak interaction.
The eight gluons mediate the strong interactions between color charged particles (the quarks). Gluons are massless. The eightfold multiplicity of gluons is labeled by a combination of color and anticolor charge. Because the gluons have an effective color charge, they can also interact among themselves. The gluons and their interactions are described by the theory of quantum chromodynamics.
The interactions between all the particles described by the Standard Model are summarized by the diagrams on the right of this section.

The Higgs particle is a massive scalar elementary particle theorized by Peter Higgs in 1964, when he showed that Goldstone's 1962 theorem (generic continuous symmetry, which is spontaneously broken) provides a third polarisation of a massive vector field. Hence, Goldstone's original scalar doublet, the massive spin-zero particle, was proposed as the Higgs boson. (see 1964 PRL symmetry breaking papers) and is a key building block in the Standard Model. It has no intrinsic spin, and for that reason is classified as a boson (like the gauge bosons, which have integer spin).\\ \\
The Higgs boson plays a unique role in the Standard Model, by explaining why the other elementary particles, except the photon and gluon, are massive. In particular, the Higgs boson explains why the photon has no mass, while the W and Z bosons are very heavy. Elementary-particle masses, and the differences between electromagnetism (mediated by the photon) and the weak force (mediated by the W and Z bosons), are critical to many aspects of the structure of microscopic (and hence macroscopic) matter. In electroweak theory, the Higgs boson generates the masses of the leptons (electron, muon, and tau) and quarks. As the Higgs boson is massive, it must interact with itself.\\ \\
Because the Higgs boson is a very massive particle and also decays almost immediately when created, only a very high-energy particle accelerator can observe and record it. Experiments to confirm and determine the nature of the Higgs boson using the Large Hadron Collider (LHC) at CERN began in early 2010 and were performed at Fermilab's Tevatron until its closure in late 2011. Mathematical consistency of the Standard Model requires that any mechanism capable of generating the masses of elementary particles becomes visible at energies above 1.4 TeV; therefore, the LHC (designed to collide two 7 TeV proton beams) was built to answer the question of whether the Higgs boson actually exists.\\ \\
On 4 July 2012, two of the experiments at the LHC (ATLAS and CMS) both reported independently that they found a new particle with a mass of about 125 GeV/c2, which is ''consistent with the Higgs boson''. It was later confirmed to be the searched-for Higgs boson.

\section{The Top Quark}
\label{sec:theory_top}
