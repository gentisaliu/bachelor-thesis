This chapter discusses the relevant theory for the thesis. Section \ref{sec:theory_sm} gives a condensed overview of the experimental aspect of the standard model (SM) of particle physics, the framework for this work. Section \ref{sec:theory_top} focuses on properties of the top quark, its production processes and decay modes, which will be relevant for the development and evaluation of the classifier in chapters \ref{ch:classifier} and \ref{ch:classifier}.
\section{The Standard Model of Particle Physics}
\label{sec:theory_sm}
The SM describes the known elementary particles and their interactions. Elementary particles are grouped into fermions and bosons based on their spin quantum number.\\ \\
\textbf{Fermions} are particles with a spin of $\nicefrac{1}{2}$. They are grouped into six quarks and six leptons, which are arranged according to their masses in three generations. The quarks are the up ($\textrm{u}$), down ($\textrm{d}$), charm ($\textrm{c}$), strange ($\textrm{s}$), top ($\textrm{t}$) and bottom ($\textrm{b}$) quark. The leptons consist of the electron ($\textrm{e}$), electron neutrino ($\upnu_\textrm{e}$), muon ($\upmu$), muon neutrino ($\upnu_\upmu$), tau ($\uptau$), tau neutrino ($\upnu_\uptau$). For each fermion an antifermion with same mass, but opposite electric charge, color and third component of isospin, exists. A summary of the fermions is given in table \ref{tab:ch_1_sm_fermions}.

\begin{table}[h]
    \caption[Fermions of the Standard Model]{Six leptons and six quarks constitute the fermions of the Standard Model. Particles are ordered according to their generation. Source: \cite{povh,schenkel}}
    \label{tab:ch_1_sm_fermions}
    \begin{tabular}{ccccccccc}
        \toprule
        \multirow{2}{*}{Fermions} & \multicolumn{3}{c}{Generation} & {Electric} & \multirow{2}{*}{Color} & {$3^{\mathrm{rd}}$ component} & \multirow{2}{*}{Spin}\\
        & 1 & 2 & 3 & {charge} & & {of isospin $I_3$} & \\
        \midrule
        \multirow{2}{*}{Leptons} & $\upnu_\textrm{e}$ & $\upnu_\upmu$ & $\upnu_\uptau$ & {$0$} & \multirow{2}{*}{---} & {$+\nicefrac{1}{2}$} & \multirow{2}{*}{$\nicefrac{1}{2}$}\\
        & e & $\upmu$ & $\uptau$ & {$-1$} & & {$-\nicefrac{1}{2}$} &\\
        \midrule
        \multirow{2}{*}{Quarks} & u & c & t & $+\nicefrac{2}{3}$ & \multirow{2}{*}{r, b, g} & {$+\nicefrac{1}{2}$} & \multirow{2}{*}{$\nicefrac{1}{2}$}\\
        & d & s & b & $-\nicefrac{1}{3}$ & & {$-\nicefrac{1}{2}$} & \\
        \bottomrule
    \end{tabular}
\end{table}

Quarks carry color charge, electric charge and weak isospin - they interact with other quarks via the strong, electromagnetic and weak interaction respectively. A quark's color can take one of three charges (red, green, and blue), an antiquark one of the three anticolors (antired, antigreen, and antiblue). Due to color confinement, quarks cannot be isolated, they are strongly bound to one another forming color-neutral composite particles (hadrons). Hadrons consist of mesons (one quark, one antiquark) and baryons (three quarks), such as the proton (uud) and the neutron (udd). Quarks from the same generation form a weak isospin doublet; particles from the same doublet behave similarly towards the weak interaction.

Leptons do not possess color charge. The three neutrinos lack additionally in electric charge and only interact through the weak nuclear force, thus being difficult to detect. The electron, muon, and tau have electric charge and interact electromagnetically.

First-generation charged particles do not decay, they constitute ordinary matter. Neutrinos do not decay either, they exist in abundance, but do not interact with matter. Other higher generation charged particles have very short lifetimes and can be observed in high-energy experiments only \cite{wiki:standardmodel}.\\

Gauge \textbf{bosons} mediate the forces between elementary particles and have a spin of 1. They consist of the photon $\upgamma$, mediating the electromagnetic interaction between electrically charged particles, the $W^{+}$, $W^{-}$ and $Z$ bosons mediating the weak interaction and the 8 gluons $g$, mediating the strong interaction. They are listed in detail in table \ref{tab:ch_1_sm_bosons}.\\

\begin{table}[h]
	\caption[Gauge bosons of the Standard Model]{The gauge bosons of the Standard Model are spin-1 particles that mediate the three fundamental forces described by the SM. For every boson, the interaction it participates on, the charge it couples on, its mass, spin, parity and interaction range are given. Source: \cite{povh}, \cite{faltermann}}
	\label{tab:ch_1_sm_bosons}
	\begin{tabular}{cccccc}
		\toprule
		Boson & Interaction & Acts on & Mass (\SI[parse-numbers = false]{\giga\eV}) & {$J^P$} & Range (\SI[parse-numbers = false]{\meter})\\
		\midrule
		8 gluons (g) & strong & color charge & {$0$} & {$1^-$} & $\approx 10^{-15}$\\
		photon ($\upgamma$) & electromagnetic & electric charge & {$0$} & {$1^-$} & {$\infty$}\\
		W$^{\pm}$ & \multirow{2}{*}{weak} &\multirow{2}{*}{weak charge} & {$80.385$} & \multirow{2}{*}{$1$} & \multirow{2}{*}{$10^{-18}$}\\
		Z$^0$ & & & {$91.188$} & &\\
		\bottomrule
	\end{tabular}
\end{table}

The interactions of the SM are described by quantum field theories \cite{welsch}:
\begin{itemize}
\item the electromagnetic interaction is described by quantum electrodynamics (QED). Because photons have no electrical charge, they do not interact with other photons,
\item the strong interaction is described by quantum chromodynamics (QCD), which, as previously mentioned, forbids the existence of free color-charged particles,
\item the weak interaction is described by quantumflavordynamics (QFD) \cite{griffiths}. The gauge bosons, left-chiral fermions and right-chiral antifermions interact weakly. Conservation of parity is therefore violated by the weak interaction, because left-handed particles transform to right-handed particles under a parity inversion.
\end{itemize}

The electromagnetic and the weak interaction are unified within the theory of the electroweak interaction. This theory predicts four massless gauge bosons, the B$^{0}$, W$^{0}$, W$^{1}$, W$^{2}$ bosons. The spontaneous symmetry breaking of the electroweak symmetry, caused by the Higgs mechanism, produces three massive gauge bosons (the Z$^0$ and W$^{\pm}$) and one massless (the photon $\upgamma$). These gauge bosons are described by the theory as orthogonal linear combinations of the B$^{0}$ and W$^{0}$ bosons under the Weinberg angle \cite{wiki:electroweak}.\\

The Higgs mechanism introduces a new complex scalar Higgs field, which in addition to providing masses to the weak gauge bosons, also couples through a Yukawa interaction with the corresponding fermion fields, generating non-vanishing fermion mass terms. The excitation of the Higgs field leads to the production of the Higgs boson. The existence of the Higgs boson was confirmed in 2012 by the ATLAS and CMS experiments at the Large Hadron Collider. It has a mass of around \SI{125}{\giga\eV} \cite{chatrchyan} and is the only spin-0 particle in the SM.

\section{The Top Quark}
\label{sec:theory_top}

\begin{figure}[H]
    \unitlength = 1mm
    \begin{fmffile}{simple_labels}
    \begin{fmfgraph*}(40,25)
    \fmfleft{i1,i2}
    \fmfright{o1,o2}
    \fmflabel{$e^-$}{i1}
    \fmflabel{$e^+$}{i2}
    \fmflabel{$e^+,\mu^+$}{o1}
    \fmflabel{$e^-,\mu^-$}{o2}
    \fmflabel{$i\sqrt{\alpha}$}{v1}
    \fmflabel{$i\sqrt{\alpha}$}{v2}
    \fmf{fermion}{i1,v1,i2}
    \fmf{fermion}{o1,v2,o2}
    \fmf{photon,label=$\gamma,,Z^0$}{v1,v2}
    \end{fmfgraph*}
    \end{fmffile}
    %
    \hfill
    %
    \begin{fmffile}{box}
    \begin{fmfgraph}(40,15)
    \fmfleft{i1,i2}
    \fmfright{o1,o2}
    \fmf{fermion}{i1,v1,v2,o1}
    \fmf{fermion}{o2,v4,v3,i2}
    \fmf{photon,tension=0}{v1,v3}
    \fmf{photon,tension=0}{v2,v4}
    \end{fmfgraph}
    \end{fmffile}
    
    \medskip
    \caption{Using \texttt{feynmp}}
\end{figure}
