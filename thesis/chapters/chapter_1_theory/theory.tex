\chapter{The Standard Model}
\label{ch:theory}

The standard model of particle physics (SM) describes all known elementary particles and three of the four known fundamental forces. These forces are the electromagnetic interaction, the weak interaction and the strong interaction. Gravity is not included yet, but is negligible on a microscopic scale. The SM predicted many particles that were found later in experiments, for example the top quark in 1995 \cite{tdiscovery}, the tau neutrino in 2000 \cite{tau2000} or the Higgs boson in 2012 \cite{ATLAS_higgs_1207, CMS_higgs_1207}. It also provides precise predictions about the properties of elementary particles like the Landé $g$-factor \cite{Lande}. Despite the great success of the SM, hints exist that the particles and their interactions are described by a more fundamental mechanism. For example, many free parameters, like the masses of particles, cannot be predicted by the SM and have to be determined by measurements. Furthermore, dark matter and dark energy are not described by the SM. Therefore, to find evidences for new physics, it is important to measure the particle properties predicted by the SM with best possible precision. \\

This chapter gives a brief summary of the SM with focus on the electroweak interaction and quantum chromodynamics. A more detailed description of the SM can be found in references \cite{QuarksAndLeptons, Gordon, PhysicsFromSymmetry, Peskin}. Natural units with $\hbar = c = 1$ are used to simplify the formulas quoted in this chapter. 


\section{Overview}

The SM is a quantum field theory that describes the particles as fields and that includes quantum mechanics as well as special relativity. It is a gauge theory in which underlying interactions are described by the $\textrm{U}(1)_\textrm{Y}\otimes \textrm{SU}(2)_\textrm{L}\otimes \textrm{SU}(3)_\textrm{C}$ symmetry group. \\

Special relativity requires a Lorentz invariant description of the SM. The spin of a particle defines its representation of the Lorentz group and has a direct influence on its behavior under Lorentz transformation. With respect to the spin quantum number, the particles in the SM can be divided into two groups, fermions with half integer spin and bosons with integer spin. Fermions are described as spinor fields and are the constituents of all microscopic matter, for example atoms. They exist in three generations, where only the first generation of particles is stable and does not decay over time. Fermions can be further distinguished into quarks and leptons. Quarks are the constituents of protons and neutrons, they interact through all forces and can be distinguished by their electric charge. Quarks with an electric charge of $+\nicefrac{2}{3}\,\si{\elementarycharge}$ are called up-type quarks, while down-type quarks carry an electric charge of $-\nicefrac{1}{3}\,\si{\elementarycharge}$. The charged leptons, the electron, the muon and the tauon can interact via the electromagnetic and weak force. The neutral leptons, the neutrinos, can only interact via the weak force. A summary of the fermions is given in table \ref{tab:ch_1_fermions}. The second group of the SM consists of bosons. Bosons with spin 1, described by vector fields, are the mediator particles for the three fundamental forces of the SM: the photon ($\upgamma$) for the electromagnetic interaction, the $\textrm{W}^{\pm}$ and Z bosons for the weak interaction and the gluons (g) for the strong interaction. The Higgs boson is the only particle of the SM with spin 0 and is described by a scalar field. It interacts with all massive particles. A summary of the bosons is shown in table \ref{tab:ch_1_bosons}.\\

\begin{table}[t]
\caption[Fermions of the SM]{\textbf{The fermions of the standard model.} All quarks and leptons of the SM and their electroweak charges are listed. The fermions occur with negative chirality (left handed) in weak isospin doublets or with positive chirality (right handed) in weak isospin singlets. Under the assumtion that the neutrino mass is zero, there is no right handed neutrino in the SM. For each fermion, an antifermion exists with opposite charge and chirality.}
\label{tab:ch_1_fermions}
\begin{tabular}{lcccSSS}
\toprule
\multirow{2}{*}{Fermions} & \multicolumn{3}{c}{Generation} & {Weak hyper- } & {$3^{\mathrm{rd}}$ comp.} & {Electric} \\ 
& 1 & 2 & 3 &  {charge $Y_\textrm{W}$} &  {of isospin $I_3$} & {charge $Q$} \\
\midrule
\multirow{4}{*}{Quarks} 
	& \multirow{2}{*}{$\myvec{\textrm{u}\\\textrm{d}}_\textrm{L}$} 
	& \multirow{2}{*}{$\myvec{\textrm{c}\\\textrm{s}}_\textrm{L}$} 
	& \multirow{2}{*}{$\myvec{\textrm{t}\\\textrm{b}}_\textrm{L}$} 
	& {$+\nicefrac{1}{3}$} & {$+\nicefrac{1}{2}$} & {$+\nicefrac{2}{3}$}\\
	& & & & {$+\nicefrac{1}{3}$} & {$-\nicefrac{1}{2}$} & {$-\nicefrac{1}{3}$}\\
	& {$\textrm{u}_\mathrm{R}$} & {$\textrm{c}_\mathrm{R}$} & {$\textrm{t}_\mathrm{R}$} & {$+\nicefrac{4}{3}$} & 0 & {$+\nicefrac{2}{3}$}\\
	& $\textrm{d}_\mathrm{R}$ & {$\textrm{s}_\mathrm{R}$} & {$\textrm{b}_\mathrm{R}$} & {$-\nicefrac{2}{3}$} & 0 & {$-\nicefrac{1}{3}$}\\
\midrule
\multirow{3}{*}{Leptons}
	& \multirow{2}{*}{$\myvec{\upnu_\textrm{e}\\\textrm{e}}_\textrm{L}$} 
	& \multirow{2}{*}{$\myvec{\upnu_\upmu\\ \upmu}_\textrm{L}$} 
	& \multirow{2}{*}{$\myvec{\upnu_\uptau\\ \uptau}_\textrm{L}$} 	 
	& {$-1$} & {$+\nicefrac{1}{2}$} & {$0$}\\
	& & & & {$-1$} & {$-\nicefrac{1}{2}$} & {$-1$}\\
	& {$\textrm{e}_\mathrm{R}$} & {$\upmu_\mathrm{R}$} & {$\uptau_\mathrm{R}$} & {$-2$} & {$0$} & {$-1$}\\
\bottomrule
\end{tabular}
\end{table}

\section{Electroweak Interaction}

The electromagnetic interaction and the weak interaction are described in a unified electroweak theory by the $\textrm{U}(1)_\textrm{Y} \otimes \textrm{SU}(2)_\textrm{L}$ group. By demanding the theory to be locally invariant under $\textrm{U}(1)_\textrm{Y}$ transformations, there must exist a massless vector boson ($\textrm{B}$), and a conserved quantity, the weak hypercharge $Y_\textrm{W}$. To obtain a local $\textrm{SU}(2)_\textrm{L}$ symmetry, three additional massless vector bosons ($\textrm{W}_0$, $\textrm{W}_1$ and $\textrm{W}_2$) are needed; the conserved quantity of this symmetry is the third component of the weak isospin $I_3$. The electric charge
\begin{equation}
Q = \frac{Y_\textrm{W}}{2} + I_3 \quad ,
\end{equation}
is related with these conserved quantities and is hence conserved. The Higgs boson originates from a spontaneous symmetry breaking of the local $\textrm{U}(1)_\textrm{Y} \otimes \textrm{SU}(2)_\textrm{L}$ symmetry and leads to a non-diagonal mass matrix for the four vector bosons. This mass matrix can be diagonalized to find the mass eigenstates 
\begin{equation}
\begin{split}
\upgamma =& \cos(\theta_\textrm{W}) \textrm{B} + \sin(\theta_\textrm{W}) \textrm{W}_0 \quad ,	\\
Z =& - \sin(\theta_\textrm{W}) \textrm{B} + \cos(\theta_\textrm{W}) \textrm{W}_0 \quad ,		\\
\textrm{W}^{\pm} =& \frac{1}{\sqrt{2}}(W_1 \mp \textrm{i} \textrm{W}_2) \quad ,
\end{split}
\end{equation}
with the Weinberg angle $\theta_\textrm{W}$. In these mass eigenstates, the photon remains massless while the Z, $\textrm{W}^+$ and $\textrm{W}^-$ bosons receive a mass. The Z boson and the photon carry no electric charge, while the $\textrm{W}^\pm$ bosons carry an electric charge of $\pm 1\,\si{\elementarycharge}$. The large mass of the Z and $\textrm{W}^\pm$ bosons is responsible for the short range of the weak interaction, as they have a lifetime of less than $10^{-24}$\,s \cite{pdg}. The effective strength can be estimated with the Fermi coupling constant of $G_\textrm{F} \approx 10^{-5}\,\GeV^{-2}$ of the weak interaction. The electromagnetic coupling constant has a value in the order of $10^{-2}$.\\

\begin{table}
\caption[Bosons of the standard model]{\textbf{Summary of the bosons of the SM.} The vector bosons with spin 1, the photon, the W$^{\pm}$ and Z bosons and the gluons are the mediator particles of the three fundamental forces that are described by the SM. The Higgs boson with spin 0 is not mediating a force, but it couples to the mass of the particles, masses are taken from~\cite{pdg}}
\label{tab:ch_1_bosons}
\begin{tabular}{lllS}
\toprule
Boson & Force & Couples to & {Mass (GeV)} \\
\midrule
photon ($\upgamma$)& electromagnetic & electric charge & 0  \\
W$^{\pm}$ & \multirow{2}{*}{weak} & \multirow{2}{*}{weak charge}  & 80.385 \\
Z & & &  91.188  \\
8 gluons (g) &  strong & color charge  & 0  \\
Higgs (H) & - & mass & 125.09 \\
\bottomrule
\end{tabular}
\end{table}

With respect to $\textrm{SU}(2)_\textrm{L}$ transformations, the quarks and leptons of one generation are grouped into weak isospin doublets, consisting of two left-handed spinor fields with $I_3 = \pm \nicefrac{1}{2}$. Weak isospin singlets with $I_3 = 0$ are represented by right-handed spinor fields, which cannot couple to charged currents. By interacting through $\textrm{W}^\pm$ bosons, flavor\footnote{The flavor indicates the type of a particle. For example, there are six quark flavors (u, d, s, c, b, t) and six lepton flavors (e, $\upmu$, $\uptau$ , $\upnu_\textrm{e}$, $\upnu_\upmu$, $\upnu_\uptau$).} changing charged currents are possible. They describe transitions of fermions within a weak isospin doublet, for example $\textrm{u} \rightarrow \textrm{d} + \textrm{W}^+$. For quarks, this current is given by
\begin{equation}\label{eq:ch_1_J}
\begin{split}
J^\mu_\textrm{CC} &= \begin{pmatrix}\bar{\textrm{u}}_\textrm{} & \bar{\textrm{c}}_\textrm{} & \bar{\textrm{t}}_\textrm{}\end{pmatrix}
	\ \upgamma^\mu \ \frac{1-\upgamma^5}{2} \ \begin{pmatrix} \textrm{d'}_\textrm{}\\\textrm{s'}_\textrm{}\\\textrm{b'}_\textrm{} \end{pmatrix}\\
&= \bar{\textrm{u}}_\textrm{} \ \upgamma^\mu \ \frac{1-\upgamma^5}{2} \ \textrm{d'}_\textrm{} + \bar{\textrm{c}}_\textrm{} \  \upgamma^\mu \ \frac{1-\upgamma^5}{2} \ \textrm{s'}_\textrm{} + \bar{\textrm{t}}_\textrm{} \ \upgamma^\mu \ \frac{1-\upgamma^5}{2} \ \textrm{b'}_\textrm{} \quad ,
\end{split}
\end{equation}
with the Dirac matrices $\upgamma^\mu$ and $\upgamma^5$ and $\bar{q} = q^\dagger \gamma_0$ denoting the Dirac adjunct spinor of $q$. The quarks described in table \ref{tab:ch_1_fermions}, are in their electroweak eigenstate and are distinguished from the observable mass eigenstates. Here, $q'$ denotes a quark in its electroweak eigenstate and $q$ in its mass eigenstate. The mass eigenstates of the quarks are therefore a mixture of their electroweak eigenstates and vice versa:
\begin{equation}\label{eq:ch_1_weakmass}
\begin{pmatrix} \textrm{d}'\\\textrm{s}'\\\textrm{b}' \end{pmatrix}
= 
V \begin{pmatrix} \textrm{d}\\\textrm{s}\\\textrm{b} \end{pmatrix}
=
\begin{pmatrix} 
	V_\textrm{ud} & V_\textrm{us} & V_\textrm{ub}\\
	V_\textrm{cd} & V_\textrm{cs} & V_\textrm{cb}\\
	V_\textrm{td} & V_\textrm{ts} & V_\textrm{tb} 
\end{pmatrix}
\begin{pmatrix} \textrm{d}\\\textrm{s}\\\textrm{b} \end{pmatrix} \quad .
\end{equation}
The $3 \times 3$ unitary matrix $V$ is the Cabibbo-Kobayashi-Maskawa matrix (CKM-matrix) that allows transitions of quarks between two generations. The probability of these transitions is low due to small off-diagonal values of $V$ \cite{pdg}:
\begin{equation}\label{eq:ch_1_V}
V = 
\begin{pmatrix} 
	0.97417\pm 0.00021 & 0.2248\pm 0.0006 & 0.00409\pm 0.00039\\
	0.220\pm 0.005 & 0.995\pm 0.016 & 0.0405\pm 0.0015\\
	0.0082\pm 0.0006 & 0.0400\pm 0.0027 & 1.009\pm 0.031
\end{pmatrix} \quad .
\end{equation} 
The matrix $V$ has four free parameters that cannot be predicted by the SM and have to be determined by measurements. By inserting equation \ref{eq:ch_1_weakmass} into equation \ref{eq:ch_1_J}, one can, for instance, obtain the current for a transition of a b quark to a c quark . The relevant part is given by
\begin{equation}
J_\textrm{CC,cb}^\mu = \bar{\textrm{c}}_\textrm{}\ \upgamma^\mu\ \frac{1-\upgamma^5}{2}\ V_\textrm{cb}\ \textrm{b}_\textrm{} \quad ,
\end{equation}
which is suppressed by the factor of $V_\textrm{cb} = 0.0405$.

\section{Quantum Chromodynamics} \label{sec:ch_1_QCD}
The strong interaction is responsible for the interaction between quarks and gluons and is described by quantum chromodynamics (QCD) with the underlying summetry of the $\textrm{SU}(3)_\textrm{C}$ group. Eight gluons are needed to obtain a $\textrm{SU}(3)_\textrm{C}$ invariant description. The charge of this symmetry is called color, of which three different types exist: red, green and blue. For each color, there is also a corresponding anticolor. As this symmetry is not broken, the gluons remain massless and indistinguishable. They carry one color and one anticolor charge, this allows interactions of gluons among themselves. Quarks on the other hand carry one color charge while antiquarks are charged with an anticolor.\\

The nature of the strong force leads to a potential for a quark-antiquark pair of
\begin{equation} \label{eq:ch_1_qcdPotential}
V(r) = - a \cdot \frac{\alpha_\textrm{s}(r)}{r} + b \cdot r \quad ,
\end{equation}
with the positive constants $a$ and $b$, the radius between the two particles $r$ and the strong couplings constant $\alpha_\textrm{s}(r)$, which is dependent on the radius and is in the order of 1\footnote{For instance, at the energy of the mass of the Z boson, $M_\textrm{Z} = 91.188\, \GeV$, $\alpha_\textrm{s}(M_\textrm{Z}^2)=0.1181\pm0.0013$ \cite{pdg}}. For small radii, the first term is dominant and allows bound states. The lower the distance between two particles, the lower is the strong interaction between them, this effect is called asymptotic freedom. On the other hand, the potential increases linearly with the distance between the particles. If a quark in a bound state is pulled apart from the other quark(s) in this bound state, the energy of the system is steadily increasing. When the energy is high enough, a quark-antiquark pair is produced in order to form two color neutral objects with the separated particles. This effect is called confinement and is the reason that no color-charged isolated object exists. A color-neutral object that consists of quarks is called hadron. \\

Hadrons can be grouped into mesons and baryons. Mesons are quark-antiquark pairs, whereas baryons consist of three quarks. Besides the top quark, all quarks can be bound in hadrons, therefore a huge number of different hadrons exists. Hadrons are produced by the strong force and decay in flavor-changing charged currents as described in equation \ref{eq:ch_1_J}. Therefore, the lifetime of hadrons \cite{pdg} varies from $10^{-8}$\,s for charged pions (u$\bar{\textrm{d}}$, $\bar{\textrm{u}}$d), $10^{-12}$\,s for B mesons (u$\bar{\textrm{b}}$, $\bar{\textrm{u}}$b, $\bar{\textrm{b}}$d, b$\bar{\textrm{d}}$) to values up to $10^{-24}$\,s for hadrons in excited states like the $\uprho$ meson (u$\bar{\textrm{d}}$, $\bar{\textrm{u}}$d). The only stable hadron is the proton (uud), as the neutron (udd) is only stable in atomic nuclei. The top quark is the only quark that cannot occur in a bound state because its lifetime of about $5 \cdot 10^{-25}$\,s \cite{top_lifetime} is roughly 20 times smaller than the hadronization time.

%FIXME
%\section{Top Quark}
%pair production\\
%background\\
%decay, dileptonic, semileptonic\\
%\section{Bottom Quark}



