\chapter*{Abstract}

The top quark was the last quark in the standard model of particle physics to be discovered in 1995. It is at \SI{173}{GeV} the heaviest elementary particle in the standard model and the only quark with a mass on the same order as the electroweak symmetry breaking scale. The large mass was the cause for its late discovery, as particle accelerators with high center-of-mass energies are needed.

At the Large Hadron Collider, top quarks are predominantly produced in pairs through the strong interaction with a smaller portion being produced individually via the electroweak interaction through the $s$ channel, the $t$ channel or in the associated production with a \PW boson.

The study of $s$-channel single top quark production is important in exploring the electroweak sector of the standard model. Deviations from the standard model prediction of its cross section may hint to mechanisms for physics beyond the standard model (BSM) \cite{CMS16}. $s$-channel single top quarks account only for a small proportion of the total production of top quarks. Therefore, a good separation between signal and background events is required, with single top quarks produced via the $s$-channel constituting signal and top quark pairs (\PtopNOSPACE\APtop) background events. 

This thesis explores the improvement of the reconstruction of $s$-channel produced single top quarks over methods used so far by employing a feed-forward neural network classifier. Monte Carlo simulations for the Compact Muon Solenoid Experiment from 2017 were used to train and test the neural network. Several machine learning techniques were employed in choosing suitable input variables for the neural network and optimizing the topology of the neural network. The performance of the classifier is then compared to that of the existing reconstruction methods. It is expected that an improved reconstruction should lead to a better separation between signal and background events.