\chapter{Theoretical Background}
\label{ch:theory}

This chapter discusses the theory relevant for the thesis. Section \ref{sec:theory_sm} provides an overview of the experimental aspect of the standard model (SM) of particle physics, the framework for this work. Section \ref{sec:theory_top} focuses on properties of the top quark, its production processes and decay modes, which are important in the development and evaluation of the classifier in chapters \ref{ch:classifier} and \ref{ch:classifier}.
\section{The Standard Model of Particle Physics}
\label{sec:theory_sm}
The SM describes the known elementary particles and their interactions. Elementary particles are categorized in fermions and bosons based on their spin quantum number.

\textbf{Fermions} are particles with a spin of $\nicefrac{1}{2}$. They are grouped into six quarks and six leptons, which are arranged according to their masses in three generations. The quarks are named the up (\Pup), down (\Pdown), charm (\Pcharm), strange (\Pstrange), top (\Ptop) and bottom (\Pbottom) quark. The leptons are the electron (\Pe), electron neutrino (\Pgne), muon (\Pmu), muon neutrino (\Pgngm), tau (\Ptau), tau neutrino (\Pgngt). For each fermion an antifermion with equal mass, but opposite electric charge, color, and third component of isospin, exists. A summary of the fermions is given in table \ref{tab:ch_1_sm_fermions}.

\begin{table}[h]
    \caption[Fermions of the Standard Model]{Six leptons and six quarks constitute the fermions of the SM. Particles are ordered according to their generation. Source: \cite{Pov14,Sch17}}
    \label{tab:ch_1_sm_fermions}
    \begin{center}
        \begin{tabular}{ccccccccc}
            \toprule
            \multirow{2}{*}{Fermions} & \multicolumn{3}{c}{Generation} & {Electric} & \multirow{2}{*}{Color} & {$3^{\mathrm{rd}}$ component} & \multirow{2}{*}{Spin}\\
            & 1 & 2 & 3 & {charge} & & {of isospin $I_3$} & \\
            \midrule
            \multirow{2}{*}{Leptons} & \Pgne & \Pgngm & \Pgngt & {$0$} & \multirow{2}{*}{---} & {$+\nicefrac{1}{2}$} & \multirow{2}{*}{$\nicefrac{1}{2}$}\\
            & \Pe & \Pmu & \Ptau & {$-1$} & & {$-\nicefrac{1}{2}$} &\\
            \midrule
            \multirow{2}{*}{Quarks} & \Pup & \Pcharm & \Ptop & $+\nicefrac{2}{3}$ & \multirow{2}{*}{r, b, g} & {$+\nicefrac{1}{2}$} & \multirow{2}{*}{$\nicefrac{1}{2}$}\\
            & \Pdown & \Pstrange & \Pbottom & $-\nicefrac{1}{3}$ & & {$-\nicefrac{1}{2}$} & \\
            \bottomrule
        \end{tabular}
    \end{center}
\end{table}

Quarks carry color charge, electric charge, and weak isospin --- they interact with other quarks via the strong, electromagnetic, and weak interaction respectively. A quark's color can take one of three charges (red, green, and blue), an antiquark one of the three anticolors (antired, antigreen, and antiblue). Due to color confinement, quarks cannot be isolated, they are strongly bound to one another forming color-neutral composite particles (hadrons). Hadrons are categorized in mesons (one quark, one antiquark) and baryons (three quarks), such as the proton (\Pup\Pup\Pdown) and the neutron (\Pup\Pdown\Pdown). Quarks from the same generation form a weak isospin doublet; particles from the same doublet behave similarly towards the weak interaction.

Leptons do not possess color charge. The three neutrinos also lack charge and only interact through the weak nuclear force, thus being difficult to detect. In contrast, the electron, muon, and tau have electric charge and interact electromagnetically.

First-generation charged particles do not decay, they constitute ordinary matter. Neutrinos do not decay either, they exist in abundance, but do not interact significantly with matter. Other higher generation charged particles have short lifetimes and can be observed in high-energy experiments only \cite{wiki:standardmodel}.

Gauge \textbf{bosons} mediate the forces between elementary particles and have a spin of 1. The photon \Pgg, mediating the electromagnetic interaction between electrically charged particles, the \PWp, \PWm and \PZ bosons mediating the weak interaction and the 8 gluons \Pgluon, mediating the strong interaction are classified as gauge bosons. Their properties are listed in detail in table \ref{tab:ch_1_sm_bosons}.

\begin{table}[h]
	\caption[Gauge bosons of the SM]{The gauge bosons of the Standard Model are spin-1 particles that mediate the three fundamental forces described by the SM. For every boson, the interaction it participates in, the charge it couples to, its mass, spin, parity, and interaction range are given. Source: \cite{Pov14, Fal18, Patrignani:2016xqp}}
	\label{tab:ch_1_sm_bosons}
	\begin{center}
    	\begin{tabular}{cccccc}
    		\toprule
    		Boson & Interaction & Acts on & Mass (\SI[parse-numbers = false]{\giga\eV}) & {$J^P$} & Range (\SI[parse-numbers = false]{\meter})\\
    		\midrule
    		8 gluons (g) & strong & color charge & {$0$} & {$1^-$} & $\approx 10^{-15}$\\
    		photon (\Pphoton) & electromagnetic & electric charge & {$0$} & {$1^-$} & {$\infty$}\\
    		\PWpm & \multirow{2}{*}{weak} &\multirow{2}{*}{weak charge} & {$80.385$} & \multirow{2}{*}{$1$} & \multirow{2}{*}{$10^{-18}$}\\
    		\PZz & & & {$91.188$} & &\\
    		\bottomrule
    	\end{tabular}
	\end{center}
\end{table}

The interactions of the SM are described by quantum field theories \cite{Wel17}:
\begin{itemize}
\item The electromagnetic interaction is described by quantum electrodynamics (QED). Because photons have no electrical charge, they do not interact with other photons,
\item The strong interaction is described by quantum chromodynamics (QCD), which forbids the existence of free color-charged particles,
\item The weak interaction is described by quantum flavordynamics (QFD) \cite{Gri08}. The gauge bosons, left-chiral fermions, and right-chiral antifermions interact weakly. Conservation of parity is therefore violated by the weak interaction, because left-handed particles transform to right-handed particles under a parity inversion.
\end{itemize}

The electromagnetic and the weak interaction are unified within the theory of the electroweak interaction. This theory predicts four massless gauge bosons, the \PBzero, \PWzero, \PWone, and \PWtwo bosons. The spontaneous symmetry breaking of the electroweak symmetry, caused by the Higgs mechanism, produces three massive gauge bosons (the \PZzero and \PWpm) and one massless (the photon \Pphoton). The \PZzero and \PWpm bosons are described by the theory as orthogonal linear combinations of the \PBzero and \PWzero bosons under the Weinberg angle \cite{wiki:electroweak}.

The Higgs mechanism introduces a new complex scalar Higgs field, which provides masses for the weak gauge bosons, and couples through a Yukawa interaction with the corresponding fermion fields, generating non-vanishing fermion mass terms. The excitation of the Higgs field leads to the production of the Higgs boson. The existence of the Higgs boson was confirmed in 2012 by the ATLAS and CMS experiments at the Large Hadron Collider. It has a mass of around \SI{125}{\giga\eV} \cite{Cha12} and is the only spin-0 particle in the SM.

\section{The Top Quark}
\label{sec:theory_top}
The top quark is with $172.44 \pm 0.13 \text{ (stat)} \pm 0.47 \text{ (syst) }\SI{}{GeV}$ \cite{ACCC14} the heaviest particle in the SM. Its existence was first postulated in 1973 by Makoto Kobayashi and Toshihide Maskawa. The high mass is responsible for the top quark's short lifetime of \SI{5e-25}{s}, which is about twenty times shorter than the typical timescale of hadronization. Because no hadronization can occur, there are no composite particles made up of a top quark, making the top quark very interesting for studying ''bare'' quarks. The strong coupling between the Higgs boson and the top quark makes it a suitable tool for researching the Higgs boson and Physics Beyond the SM.

\subsection{Production and Decay}
Due to their high mass, accelerators with high energies are required for the production of top quarks. Because of this, its discovery was a long and difficult process, which began in the late seventies \cite{RevModPhys.69.137} and was concluded in 1995 at the Tevatron by the CDF and D\O{} collaborations.
\begin{figure}[H]
    \centering
    \begin{subfigure}[t]{0.2\textwidth}
        \centering
        \feynmandiagram [horizontal=a to b] {
    i1 [particle=\Pquark] -- [fermion] a[dot] -- [fermion] i2 [particle=\APquark],
    a -- [gluon, edge label=\Pgluon] b[dot],
    b -- [fermion] f1 [particle=\Ptop],
    f2 [particle=\APtop] -- [fermion] b,
};
        \caption{$q\overline{q}$ annhilation}
        \label{fig:top_pair_qqbar}
    \end{subfigure}\hfill
    \begin{subfigure}[t]{0.2\textwidth}
        \centering
        \feynmandiagram [horizontal=a to b] {
    i1 [particle=\Pgluon] -- [gluon] a[dot] -- [gluon] i2 [particle=\Pgluon],
    a -- [gluon, edge label=\Pgluon] b[dot],
    b -- [fermion] f1 [particle=\Ptop],
    f2 [particle=\APtop] -- [fermion] b,
};
        \caption{$gg$ fusion}
        \label{fig:top_pair_gg}
    \end{subfigure}\hfill
    \begin{subfigure}[t]{0.2\textwidth}
        \centering
        \feynmandiagram [small,vertical'=a to b] {
    i1 [particle=\Pgluon] -- [gluon] a[dot] -- [fermion] f1 [particle=\Ptop],
    a -- [anti fermion, edge] b[dot],
    i2 [particle=\Pgluon] -- [gluon] b -- [anti fermion] f2 [particle=\APtop],
};
        \caption{$gg$ scattering}
        \label{fig:top_pair_gg_scatter}
    \end{subfigure}\hfill
    \begin{subfigure}[t]{0.2\textwidth}
        \centering
        \begin{tikzpicture}
    \begin{feynman}
        \diagram [vertical'=a to b] {
            i1 [particle=\Pgluon] -- [gluon] a[dot] -- [draw=none] f1 [particle=\Ptop],
            a -- [fermion, edge] b[dot],
            i2 [particle=\Pgluon] -- [gluon] b -- [draw=none] f2 [particle=\APtop],
        };
        \diagram* {
            (a) -- [anti fermion] (f2),
            (b) -- [fermion] (f1),
        };
    \end{feynman}
\end{tikzpicture}

        \caption{$gg$ scattering with switched final states}
        \label{fig:top_pair_gg_scatter_switched}
    \end{subfigure}
    \caption{Top quark pair production via the strong interaction}
    \label{fig:top_pair}
\end{figure}
Top quarks are predominantly produced in pairs via the strong interaction. Figure \ref{fig:top_pair} shows the four relevant Feynman diagrams at leading order. These processes are known as $t\overline{t}$ processes. Owing to the fact that antiquarks do not appear as valence quarks in a proton, $q\overline{q}$ annihilation is unlikely to occur in proton-proton collisions, such as those in LHC experiments. Top quarks are mostly produced via gluon fusion.\\ \\
Additionally, individual (single) top quarks can be produced via the electroweak interaction. The Feynman diagrams for these processes in leading order are given in Figure \ref{fig:top_single}.
\begin{figure}[H]
    \centering
    \begin{subfigure}[t]{0.2\textwidth}
        \centering
        \feynmandiagram [horizontal=a to b] {
    i1 [particle=\APquark] -- [anti fermion] a[dot] -- [anti fermion] i2 [particle=\Pquark'],
    a -- [boson, edge label=\PWplus] b[dot],
    b -- [anti fermion] f1 [particle=\APbottom],
    f2 [particle=\Ptop] -- [anti fermion] b,
};
        \caption{s-channel}
        \label{fig:top_single_s}
    \end{subfigure}\hfill
    \begin{subfigure}[t]{0.2\textwidth}
        \centering
        \feynmandiagram [vertical'=a to b] {
    i1 [particle=\Pquark] -- [fermion] a[dot] -- [fermion] f1 [particle=\Pquark'],
    a -- [boson, edge label=\PW] b[dot],
    i2 [particle=\Pbottom] -- [fermion] b -- [fermion] f2 [particle=\Ptop],
};
        \caption{t-channel}
        \label{fig:top_single_t}
    \end{subfigure}\hfill
    \begin{subfigure}[t]{0.2\textwidth}
        \centering
        \feynmandiagram [horizontal=a to b] {
    i1 [particle=\Pgluon] -- [gluon] a[dot] -- [anti fermion] i2 [particle=\Pbottom],
    a -- [fermion, edge label=\Pbottom] b[dot],
    b -- [fermion] f1 [particle=\Ptop],
    f2 [particle=\PWminus] -- [boson] b,
};
        \caption{tW channel}
        \label{fig:top_single_tw_1}
    \end{subfigure}\hfill
    \begin{subfigure}[t]{0.2\textwidth}
        \centering
        \feynmandiagram [vertical'=a to b] {
    i1 [particle=\Pgluon] -- [gluon] a[dot] -- [fermion] f1 [particle=\Ptop],
    a -- [anti fermion, edge label=\Ptop] b[dot],
    i2 [particle=\Pbottom] -- [fermion] b -- [boson] f2 [particle=\PWminus],
};
        \caption{tW channel}
        \label{fig:top_single_tw_2}
    \end{subfigure}
    \caption{Single top quark production via the electroweak interaction}
    \label{fig:top_single}
\end{figure}
Top quarks decay almost exclusively in a \PWplus boson and a \Pbottom quark. The boson further decays either in a quark-antiquark pair or in a lepton and its corresponding neutrino. The former process is referred to as being hadronic, the latter as leptonic.
