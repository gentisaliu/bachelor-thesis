\chapter{Conclusion and Outlook}
Identifying single top quarks produced in the $s$-channel helps probe the electroweak sector of the standard model for deviations, which may hint to physics beyond the standard model. Since the cross-section for single top quark production in the $s$-channel is almost negligible compared to those of other production modes, a promising approach is to use multivariate tools to build models from standard model theory predictions and apply them to real measurements from the CMS experiment.

This thesis dealt with building a deep neural network classifier, that improved the reconstruction of the single top quark in the $s$-channel at the CMS at a center-of-mass energy of \SI{13}{TeV} in the 2j2t signal region compared to the best-mass method by over \SI{7}{\%}. Only simulation events for the year 2017 were used for training and evaluating the classifier.

This result is merely a first attempt and may be significantly improved upon in future work. First, some of the variables ranked by importance in Appendix \ref{ch:appendix_c} should be tried as classifier input variables by building upon the inputs that provided the highest improvement. Secondly, more meaningful variables may be used or derived. Thirdly, more hyperparameter optimization techniques should be tried out, i.e. random-search, genetic algorithms, breeding or Bayesian optimization. The tested grid search approach proved to be inefficient, as few hyperparameter values out of the many possibilities could be explored. A number of promising frameworks that take labeled data and generate the entire network topology with little or no supervision has been emerging and should be tried out.

This work also fell short in testing the trained classifier on measurement data by the CMS experiment as well as in testing how well the classifier performed in separating signal from background events on both simulation and data.