\chapter{Experimental Background}
\label{ch:experiment}
Since the Monte-Carlo simulations used in this thesis are modelled after collisions happening at the CMS, this chapter intends to give a brief introduction to the Large Hadron Collider (LHC) and the Compact Muon Solenoid (CMS) experiment.

\section{The Large Hadron Collider}
The Large Hadron Collider (LHC) is the world's largest and most powerful particle accelerator and is part of CERN's accelerator complex, located in the region of Geneva. It is a circular collider with a circumference of \SI{27}{km} that can accelerate protons, xenon ions and lead ions, located in a tunnel at a depth ranging from 50 to \SI{175}{m} underground.

It consists of two adjacent parallel vacuumed beam pipes, where protons circulate in opposite directions around the ring. Beams are not continuous, the protons are instead bunched together into up to 2808 bunches, each with $10^11$ protons, so that interactions can take place at discrete intervals. The beams intersect every \SI{25}{ns} at four points around the ring, where particle collisions occurr. The beam is kept by dipole magnets on its circular path, while their focus is maintained by quadrupole magnets. The magnets are kept at their operating temperature of \SI{1.9}{K} by superfluid helium-4.

The particles' energy is increased successively by the linear particle accelerator LINAC 2 at \SI{50}{MeV}, the Proton Synchroton Booster (PBS) at \SI{1.4}{GeV}, the Proton Synchroton (PS) at \SI{26}{GeV} and the Super Proton Synchroton (SPS) at \SI{450}{GeV}, where they are injected into the main ring. There, the protons are accelerated at the current energy record of \SI{6.5}{TeV} per proton and are finally circulated in the beams for 5 to 24 hours.

\begin{figure}[H]
    \centering
    \includegraphics[width=17cm]{assets/chap02/lhc.jpg}
    \caption{The accelerator complex at CERN in Switzerland and France. Source: \cite{Marcastel:1621583}}
\end{figure}

The LHC operation consists so far of two runs: in run 1 in 2010, the first collisions at a center-of-mass energy of $\sqrt{s}=\SI{3.5}{TeV}$ started. The center-of-mass energy was upgraded in 2012 to $\sqrt{s}=\SI{8}{TeV}$. After a shutdown from 2013 to 2015, Run 2 began in 2015 at $\sqrt{s}=\SI{13}{TeV}$. The LHC will shut down by the end of 2018 to enter Run 3 at $\sqrt{s}=\SI{14}{TeV}$ in 2021.

Each of the four interaction point hosts one or multiple experiments, with major ones being:
\begin{itemize}
\item \textbf{ATLAS} - A Toroidal LHC Apparatus
\item \textbf{ALICE} - A Large Ion Collider Experiment
\item \textbf{LHCb} - Large Hadron Collider beauty
\item \textbf{CMS} - Compact Muon Solenoid
\end{itemize}

The performance of a particle accelerator is quantified by the luminosity, which expresses the ratio between the number of events $N$ detected in a certain time $t$ to the interaction cross-section $\sigma$:
\begin{equation*}
    L=\frac{1}{\sigma} \dv{N}{t}
\end{equation*}
The cross section $\sigma$ is a measure for the probability of a process to occur. The larger the luminosity, the most likely an event of interest is to happen within a period of time.

\section{The Compact Muon Solenoid (CMS) Experiment}
The Compact Muon Solenoid (CMS) is a general-purpose particle detector built on the Large Hadron Collider, situated in an underground cavern at Cessy, France, built with the goal of investigating a wide range of physics, such as TeV-scale physics, the properties of the Higgs boson, heavy ion collisions, physics beyond the SM (supersymmetry, extra dimensions) and more. It has a length of \SI{28.7}{m}, a diameter of \SI{15}{m} and a weight of approximately \SI{14000}{t}. Figure \ref{fig:cms} shows a cross section view of the CMS detector.

The detector takes its name after its relatively small dimensions and weight, its myon tracking ability and powerful solenoid magnet.

The CMS is cylindrical symmetric towards the beam axis with the interaction point centrally located. The coordinate system is oriented such that the z-axis is in the direction of the beam, the polar angle $\theta$ is defined in the r-z-plane with the pseudorapidity defined as $\eta=-\ln{\tan{(\theta / 2)}}$. The momentum component transverse to the beam direction, $p_T$, is derived from the x- and y-components, the transverse energy is defined as $E_T=E \sin{\theta}$ \cite{Spannagel2017}.

\begin{figure}[H]
    \centering
    \includegraphics[width=15cm]{assets/chap02/cms.jpeg}
    \caption{Cross section view of the CMS detector and its components. Source: \cite{Davis:2205172}}
    \label{fig:cms}
\end{figure}

The inside of the detector is surrounded by a \SI{3.8}{T} magnet field, created by the solenoid magnet, orthogonal to the beam axis, which deflects all charged particles within the detector in a circular path.

The components of the CMS from the point of interaction to its extremities are as follow \cite{Bhler:48721}:
\begin{itemize}
    \item \textbf{The Silicon Tracker} is made of silicon pixels and microstrip detectors, that measure the paths of particles traveling outwards from the point of interaction. As they pass through the tracker, charged particles create electrons and holes in the semiconducting material, which lead to tiny electric signals that are amplified and detected. The charge and momentum of the particle can be inferred from the radius of the reconstructed curvature. The tracker is limited to detecting only particles with $|\eta|<2.5$. One design goal of the tracker was to keep the impact due to interactions with the tracker of the measurement on the track of the particles to a minimum.
    \item \textbf{The Electromagnetic Calorimeter} measures the energy of photons, electrons and positrons by fully absorbing them. Lead tungsten crystals are used, which, due to the absorption, move to an excited state. The crystal leaves the excited state when photons are emitted. The energy of the photons is proportional to the energy of the absorbed particle and can be measured using photodetectors.
    \item \textbf{The Hadronic Calorimeter}  measures the energy of hadrons, i.e. protons, neutrons, pions and kaons. It is made of alternating layers of absorbers and detectors, with the latter composed of plastic scintillators. The incoming particles create a cascade of secondary particles within the absorber, that cause the scintillators to emit light.
    \item \textbf{The muon detectors} detect and reconstruct the paths of muons, similar to the silicon tracker. The muon detectors are located outside the solenoid. Due to their large mass compared to electrons, muons pass the inner components of the detector with their energy loss being close to zero.
\end{itemize}

The CMS detector can not detect neutrinos, since they barely interact with matter. This is done indirectly using momentum conservation, where the sum of the transverse momenta of the collision products must vanish. The transverse momentum of the neutrinos thus equals the negative sum of the transverse momenta of the detected collision products. Assuming the neutrinos are massless, their transverse momenta equals their energy. This is also known as missing transverse energy (MET):
\begin{equation*}
\slashed{E}_T=|-\sum{\va{p_T}}|
\end{equation*}